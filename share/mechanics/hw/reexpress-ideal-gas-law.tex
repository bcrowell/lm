The ideal gas law is $PV=nkT$, and if everything is expressed in SI units, then
the Boltzmann constant $k$ has the numerical value $k=1.38\times10^{-23}\ \junit/\kunit$. You may also have seen a version of the equation in which
$n$ is expressed in units of moles rather than molecules, and the constant of
proportionality then has a different value. The following are three other
ways in which we could imagine making a different version of the equation. In
each case, explain whether this makes sense, and if so, find the value of
the constant of proportionality that would replace the SI value of $k$.\\
(a) Temperature is measured in degrees Celsius.\\
(b) The amount of gas is measured in kilograms.\\
(c) The pressure and volume are measured in the centimeter-gram-second
(cgs) system, where, for example, the unit of force is $1\ \zu{g}\unitdot\zu{cm}/\zu{s}^2$,
and all other units are based on the centimeter, gram, and second, rather than
the SI base units of meters, kilograms, and seconds.

