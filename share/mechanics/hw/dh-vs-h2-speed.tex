Most of the mass of an atom comes from its protons and neutrons. The mass of a proton is
approximately the same as the mass of a neutron. The planet Jupiter is made mostly of hydrogen
molecules.                                                                                                                                                    
A normal hydrogen molecule ($\zu{H}_2$) contains two protons (one in each atom), and no neutrons.
A small percentage of the hydrogen in the universe is in a form in which the nucleus contains
both a proton and a neutron; this is called deuterium, often notated D. Since deuterium isn't very abundant,
the most common thing to happen to a deuterium atom in Jupiter's atmosphere is that it would
find itself in a molecule whose other atom was a normal hydrogen atom. The resulting molecule
could be described as a DH. A DH molecule contains a total of two protons and one neutron.
Compare the typical speed of a DH molecule in Jupiter's atmosphere with the typical speed of
an $\zu{H}_2$. Give a quantitative comparison, and notate it so that it's clear which is the higher
speed.
