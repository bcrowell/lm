Although this is not a plane wave, if we take any small section of it, such
as one of the squares in the figure, it can be approximated as a plane wave.
Therefore we expect the electric and magnetic fields to be like those in
a plane wave: perpendicular to each other and with $E=cB$.
Since they are perpendicular to each other, the cross product occurring
in the expression for the Poynting vector is equal to the product of the
magnitudes $EB$, and we must have $EB\propto r^{-2}$. Because $E=cB$, the
two fields must have the same dependence on $r$, and this means that we
must have both $E\propto r^{-1}$ and $B\propto r^{-1}$. This is somewhat
counterintuitive; it tells us that radiation fields fall off \emph{more
slowly} than the static field of a point source.
