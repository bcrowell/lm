Assume that the kinetic energy of an electron in the $n=1$
state of a hydrogen atom is on the same order of magnitude
as the absolute value of its total energy, and estimate a
typical speed at which it would be moving. (It cannot really
have a single, definite speed, because its kinetic and
interaction energy trade off at different distances from the
proton, but this is just a rough estimate of a typical
speed.) Based on this speed, were we justified in assuming
that the electron could be described nonrelativistically?
