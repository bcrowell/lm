\epigraphlong{[In] a few minutes I shall be all melted... I have been
wicked in my day, but I never thought a little girl like you
would ever be able to melt me and end my wicked deeds.
Look out --- here I go!}{The \index{Wicked Witch of the West}Wicked Witch
 of the West}

As the Wicked Witch learned the hard way, losing molecular
cohesion can be unpleasant. That's why we should be very
grateful that the concepts of quantum physics apply to
matter as well as light. If matter obeyed the laws of
classical physics, \index{molecules!nonexistence in
classical physics}molecules wouldn't exist.

Consider, for example, the simplest atom, hydrogen. Why does
one hydrogen atom form a chemical bond with another hydrogen
atom? Roughly speaking, we'd expect a neighboring pair of
hydrogen atoms, A and B, to exert no force on each other
at all, attractive or repulsive: there are two repulsive
interactions (proton A with proton B and electron A with
electron B) and two attractive interactions (proton A with
electron B and electron A with proton B). Thinking a
little more precisely, we should even expect that once the
two atoms got close enough, the interaction would be
repulsive. For instance, if you squeezed them so close
together that the two protons were almost on top of each
other, there would be a tremendously strong repulsion
between them due to the $1/r^2$ nature of the electrical
force. A more detailed calculation using classical physics
gives an extremely weak binding, about 1/17 the strength of
what we actually m4_ifelse(__mod,1,[:observe (\note{h2-classical}),:],[:observe,:]) 
which is far too weak to make the bond hold together.

Quantum physics to the rescue! As we'll see shortly, the
whole problem is solved by applying the same quantum
concepts to electrons that we have already used for photons.
