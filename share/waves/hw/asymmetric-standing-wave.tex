Consider a one-dimensional standing-wave pattern that has asymmetric boundary
conditions at the ends, i.e., for whatever measure of amplitude we have chosen,
the amplitude is zero at one end but has an extremum at the other end.
The purpose of this problem is to predict the frequencies of vibration, as
we did in problem \ref{hw:symmetric-standing-wave}, p.~\pageref{hw:symmetric-standing-wave},
for the symmetric case.\\
(a) Sketch the first three patterns. Check yourself against figure
figure \subfigref{lowest-modes-of-air-column}{3} for the first pattern and
the answer to self-check \ref{sc:fish-pattern} for the second one.\hwendpart
(b) As a warm-up with concrete numbers, consider the case where the length $L$ is $1\ \munit$.
Find the wavelengths of the patterns you drew in part a.\hwendpart
(c) Returning to the general case where $L$ is a variable, find the pattern of wavelengths.
You can do this either by writing a list ending in ``\ldots'' that clearly shows the pattern,
or by defining an $N$ and writing an equation in terms of that $N$. There is more than one
way to define an $N$, so if you do that, explain what your definition is and what are the
permissible values of $N$.\hwendpart
(d) Let $v$ be the speed of the waves. Make a prediction of the frequencies in a similar style. Check
against the answer to self-check \ref{sc:fish-pattern}.\hwendpart
(e) The clarinet acts as an asymmetric air column.
Its lowest note is produced by closing all the tone holes. The note produced
in this way has a fundamental frequency (lowest harmonic) of $147\ \zu{Hz}$, but this tone also contains
all the frequencies predicted in part d. Predict the frequency
of the third harmonic, i.e., the third-longest wavelength.\answercheck\hwendpart
