(a) In example \ref{eg:spherical-capacitor-field-given-voltage},
p.~\pageref{eg:spherical-capacitor-field-given-voltage},
we analyzed a spherical capacitor. Building on that analysis,
let's find its capacitance.
There are two ways that might occur to us to approach this,
using either one of the relations $U=q^2/2C$ and $V=q/C$.
Complete the calculation by whichever method seems like it
will be easier.\answercheck\hwendpart
(b) In the limit where the gap between $a$ and $b$ is very small,
show that you can recover the result of example \ref{eg:capparplate}, p.~\pageref{eg:capparplate}, for
a parallel-plate capacitor, i.e., the curvature doesn't matter in this limit.\hwendpart
(c) Find the capacitance of the surface of the
earth, assuming there is an outer spherical ``plate'' at infinity.
(In reality, this outer plate would just represent some distant
part of the universe to which we carried away some of the earth's charge
in order to charge up the earth.)\answercheck

