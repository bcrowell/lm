$N$ identical gears of radius $r$ are arranged with their axes parallel and coplanar. The figure shows
the $N=3$ case as an example. Each gear is an insulator, and has charge $q$
distributed uniformly about its circumference. If the system spins at frequency $f$,
find the total dipole moment. How is this different from an example like the one
in figure \subfigref{potato-chip}{3}, p.~\pageref{fig:potato-chip}? <% hw_hint("dipole-gears") %>
\hwremark{This is not an unreasonable model of the magnetic properties of a linear molecule, if
the magnetic interactions are like the ones described in problem \ref{hw:dipole-dipole-energy}.}
