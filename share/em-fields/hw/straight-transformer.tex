The thick line in the figure is a long, straight wire carrying an alternating current with amplitude I and frequency $f$.
The wiring drawn with a thin line is a small coil consisting of $n$ loops of wire, each loop being a square with sides of length $a$.
These loops are
bundled together and lie flat in the plane of the page (i.e., this is not a solenoid). 
The driving current in the straight wire induces a voltage in the square coil.
The twisted pair of wires connects the coil to an oscilloscope, which measures the amplitude $V$ of the voltage in
this secondary circuit. As described in example \ref{eg:twisted-pair-pickup},
p.~\pageref{eg:twisted-pair-pickup}, there is negligible voltage induced in the twisted-pair cable.
The center of the coil is at a distance $r$ from the wire, and because $a$ is small compared
to $r$, it is valid to approximate the field within the entire secondary using the field at its center.
Find $V$ in terms of the other variables.
