This problem will lead you through the steps of applying the Biot-Savart law to prove that the
magnetic field of a long, straight wire has magnitude
\begin{equation*}
  B = \frac{2 kI}{ c^2 R}.
\end{equation*}
Almost everything in this equation has to be the way it is because of units, the only
exception being the unitless factor of 2, so this problem amounts to proving that it
really does come out to be 2%
m4_ifelse(__problems,1,[.],[:,
  a fact that we previously proved in example \ref{eg:b-wire} on page
  \pageref{eg:b-wire}, using relativity.\index{magnetic field!long, straight wire!found using Biot-Savart law}
:])
(a) Set up the integral prescribed by the Biot-Savart law, and simplify it so that it involves
only scalar variables rather than a vector cross product, but do not evaluate it yet.\\
(b) Your integral will contain several different variables, each of which is changing as
we integrate along the wire. These will probably include a position on the wire,
a distance from the point on the wire to the point at which the field is to be found,
and an angle between the wire and this point-to-point line. In order to evaluate the integral,
it is necessary to express the integral in terms of only one of these variables. It's not obvious,
but the integral turns out to be easiest to evaluate if you express it in terms of the angles
and eliminate the other variables. Do so. Note that the $\der\ldots$ part of the integral has
to be reexpressed in the same way we would do any time we attacked an integral by substitution
(``$u$-substitution'').\\
(c) Pulling out all constant factors now gives a definite integral. Evaluate this integral,
which you should find is a trivial one, and show that it equals 2.


